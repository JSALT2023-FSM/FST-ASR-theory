\section{What is an ASR ``model''?}
\label{sec:asr}

Suppose we have a speech input $\mathbf{X}\in \mathbb{R}^{T^{\prime}\times F}$. 
%
We have some reference transcript $\mathbf{w} = (w_1,\ldots,w_L)$, where $w_l \in \Sigma$, and $\Sigma$ is the vocabulary. 
%
The goal of ASR is to predict $\hat{\mathbf{w}}$, which is the best estimate for $\mathbf{w}$. 
%
This is usually done by modeling $P(\mathbf{w}|\mathbf{X})$ (either directly or indirectly using Bayes rule) and computing the Bayes optimal solution (exact or approximate), i.e., 

\begin{equation}
\label{eq:asr}
    \hat{\mathbf{w}} = \text{arg}\max_{\mathbf{w}} P(\mathbf{w}|\mathbf{X})
\end{equation}

\paragraph{} Most complications in the ASR task arise from the fact that $T'$ is not equal to $L$, and there is no known alignment between the input and output sequence. 
%
Indeed, if an alignment is known, we could simply use frame-level cross-entropy for training, as was standard in older hybrid models before sequence-level objectives became popular.

\paragraph{} When we talk about ASR ``models,'' we are usually referring to a framework comprising several components, which we will discuss next.

\subsection{Frame-synchronous and label-synchronous models}

Based on how the input is processed, ASR models can be categorized as frame-synchronous or label-synchronous~\cite{Prabhavalkar2017ACO, Dong2020ACO}. 
%
Frame-synchronous models are driven by the input frames: they process the frames one-at-a-time and stop when there are no more frames to process. 
%
Some examples are traditional HMM-based models, CTC, RNN-transducers, etc. Label-synchronous models, on the other hand, are driven by the output text. 
%
Often they process the entire input and then start predicting output tokens one-by-one until an \texttt{<eos>} token is predicted. 
%
The dominant label-synchronous model is the attention-based encoder-decoder. 
%
In this document, our focus is on frame-synchronous models since they have several obvious advantages:

\begin{enumerate}
    \item They are naturally streaming if the encoder is left-context only.
    \item In addition to the tokens, they provide a sense of ``alignment'' between the label and the frames, which can be used to estimate token-level timestamps.
\end{enumerate}

\subsection{Network architecture}

Modern ASR systems usually process the input $\mathbf{X}$ using a neural network function $h(\mathbf{X}) = \mathbf{h} \in \mathbb{R}^{T\times D}$, where $T \leq T'$ usually due to frame sub-sampling, and so equation (\ref{eq:asr}) becomes
%
\begin{equation}
    \hat{\mathbf{w}} = \text{arg}\max_{\mathbf{w}} P(\mathbf{w}|\mathbf{h}).
\end{equation}

Here, $h(\cdot)$ may be the functional equivalent of any standard architecture, such as convolutional, recurrent, or self-attention layers. 
%
Often, the encoder architecture itself may also be referred to as the ``model'' by some abuse of notation. 
%
For the purpose of this document, the specific choice of architecture is not important.

\paragraph{} In some cases, the output label sequence may also be processed by a neural network to provide additional context and make the outputs conditionally dependent, such as in RNN-Transducers through a prediction network. 
%
However, this is beyond the scope for this document.

\subsection{Modeling units}

The final transcript that we want to generate using an ASR system would usually contain words. 
%
However, it is often computationally easier for the model to predict the output in the form of sub-words units. 
%
The most common modeling units used are as follows:

\begin{enumerate}
    \item Characters
    \item Phonemes
    \item Character groups, such as bi-chars, tri-chars, etc.
    \item Phoneme groups, such as bi-phones, tri-phones, etc.
    \item Chenones/senones, which are context dependent char/phoneme groups tied with a decision tree
    \item BPE units
\end{enumerate}

\paragraph{} Due to historical reasons (e.g., where the models were developed), CTC models developed using character/BPE units while LF-MMI used senones. 
%
But the choice of modeling unit is not tied to the model, and different units may be used based on their advantages/disadvantages~\cite{zhang_lattice-free_2021}. 
%
For example, using phone-based modeling units requires a phonetic lexicon, and using chenones/senones would require state tying using likelihoods from a GMM. 
%
Modern ``end-to-end'' ASR models have converged to the choice of BPE as the modeling units.

\paragraph{} Nevertheless, in the WFST-based decoding framework, the decoding graph can be built to directly output words even if the model predicts sub-word units. 
%
This is done by composing the graph with the lexicon FST $L$.


\subsection{Label topology}

Earlier, we mentioned that usually when training ASR models, the exact ``alignment'' between $\mathbf{X}$ and $\mathbf{y}$ is not known. 
%
The label topology is the component which defines the mapping between the label sequence and the modeling units. 
%
Let us denote this as $\mathcal{T}$. 
%
Consider also the linear FST denoting the label sequence as $\mathcal{Y}(\mathbf{w})$\footnote{We assume here that $\mathbf{w}$ has been tokenized into the desired modeling units.}. 
%
Then, the ``alignment lattice,'' i.e., the set of all possible alignments mapping to $\mathbf{w}$ is given by the composition of $\mathcal{T}$ and $\mathcal{Y}(\mathbf{w})$, i.e.,

\begin{equation}
    \mathcal{A}(\mathbf{w}) = \mathcal{T} \circ \mathcal{Y}(\mathbf{w}).
\end{equation}

Suppose our vocabulary consists of the characters $a$ and $b$. 
%
For traditional HMM-based systems which also modeled $\sil$ explicitly, a 1-state HMM topology can be represented by Fig.~\ref{fig:hmm_topo}.

\begin{figure}[t]
\begin{subfigure}{0.45\linewidth}
\centering
\resizebox{\textwidth}{!}{%
\begin{tikzpicture}

\node[state, style=thick] (q0) {$0$};
\node[state, above right=of q0, xshift=-0.8cm, accepting] (q1) {$1$};
\node[state, right=of q0, accepting] (q2) {$2$};

\draw 
(q1) edge[loop above] node{$\sil:\epsilon$} ()
(q1) edge[loop left] node{$a:a$} ()
(q2) edge[loop right] node{$\sil:\epsilon$} ()
(q2) edge[loop below] node{$b:b$} ()
(q0) edge[sloped, above] node{$a:a$} (q1)
(q1) edge[sloped, above, bend left] node{$b:b$} (q2)
(q2) edge[sloped, below] node{$a:a$} (q1)
(q0) edge[sloped, above] node{$b:b$} (q2);

\end{tikzpicture}
}
\caption{1-state HMM topology}
\label{fig:hmm_topo}
\end{subfigure}
\begin{subfigure}{0.55\linewidth}
\centering
\resizebox{\textwidth}{!}{%
\begin{tikzpicture}

\node[state, style=thick, accepting] (q0) {$0$};
\node[state, above right=of q0, xshift=-0.8cm, accepting] (q1) {$1$};
\node[state, right=of q0, accepting] (q2) {$2$};

\draw 
(q0) edge[loop left] node{$\phi:\epsilon$} ()
(q1) edge[loop above] node{$a:\epsilon$} ()
(q2) edge[loop right] node{$b:\epsilon$} ()
(q0) edge[sloped, above, bend left] node{$a:a$} (q1)
(q1) edge[sloped, below] node{$\phi:\epsilon$} (q0)
(q1) edge[sloped, above, bend left] node{$b:b$} (q2)
(q2) edge[sloped, below] node{$a:a$} (q1)
(q2) edge[sloped, below, bend left] node{$\phi:\epsilon$} (q0)
(q0) edge[sloped, above] node{$b:b$} (q2);

\end{tikzpicture}
}
\caption{CTC topology}
\label{fig:ctc_topo}
\end{subfigure}
\caption{Topology FSTs for standard 1-state HMM and CTC.}
\end{figure}

\paragraph{} Another popular topology in modern ASR systems is the CTC topology. 
%
The CTC topology generates alignments formed by repeating the characters and inserting blanks ($\phi$) to allow repeated characters in the label sequence. 
%
This is shown in Fig.~\ref{fig:ctc_topo}.

\subsection{Training objective}

Let $\mathcal{B}$ denote the many-to-one mapping from an alignment $\mathbf{a}$ to an output label sequence $\mathbf{w}$, which is determined by the topology $\mathcal{T}$ and the input length $T$. Then, we can expand equation (\ref{eq:asr}) using the following:

\begin{equation}
    P(\mathbf{w}|\mathbf{X}) = \sum_{\mathbf{a}\in \mathcal{B}^{-1}(\mathbf{w})} P(\mathbf{a}|\mathbf{X}).
\end{equation}

For discriminative training of the ASR model, we usually maximize the log, i.e.,

\begin{equation}
    \mathcal{L}_{\theta} = \log \left( \sum_{\mathbf{a}\in \mathcal{B}^{-1}(\mathbf{w})} P_{\theta}(\mathbf{a}|\mathbf{X}) \right).
\end{equation}

In a very general setting, the distribution can be written as:

\begin{equation}
    P_{\theta}(\mathbf{a}|\mathbf{X}) = \frac{p(\mathbf{a})\exp(f_{\theta}(\mathbf{X},\mathbf{a}))}{\mathbb{E}_{p(\mathbf{a'})}[\exp(f_{\theta}(\mathbf{X},\mathbf{a'}))]}.
\end{equation}

You may recognize that this as similar to Bayes rule on $P_{\theta}(\mathbf{a}|\mathbf{X})$, where the likelihood is replaced with a general ``potential function'' and the normalization term in the denominator is written using an expectation over all possible alignments. This way of expressing a distribution from raw (un-normalized) scores goes by many names including the Boltzmann, and Gibbs distributions. Finally, the loss function becomes:

\begin{equation}
\label{eq:loss}
    \mathcal{L}_{\theta} = \log \left( \frac{\sum_{\mathbf{a}\in \mathcal{B}^{-1}(\mathbf{w})} p(\mathbf{a})\exp(f_{\theta}(\mathbf{X},\mathbf{a}))}{\mathbb{E}_{p(\mathbf{a'})}[\exp(f_{\theta}(\mathbf{X},\mathbf{a'}))]} \right).
\end{equation}
